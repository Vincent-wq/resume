% !TEX TS-program = xelatex
% !TEX encoding = UTF-8 Unicode
% !Mode:: "TeX:UTF-8"

\documentclass{resume}
\usepackage{zh_CN-Adobefonts_external} % Simplified Chinese Support using external fonts (./fonts/zh_CN-Adobe/)
%\usepackage{zh_CN-Adobefonts_internal} % Simplified Chinese Support using system fonts
\usepackage[linkcolor=black]{hyperref} % 
\usepackage{linespacing_fix} % disable extra space before next section
\usepackage{cite}

\begin{document}
\pagenumbering{gobble} % suppress displaying page number

\name{王庆(Vincent)}

\contactInfo{\textbf{电话:} (+86) 19821243355} {\textbf{邮箱:} \href{mailto:vincent.w.qing@gmail.com}{vincent.w.qing@gmail.com}}{\textbf{GitHub:} \href{https://github.com/Vincent-wq}{https://github.com/Vincent-wq}}{}

% {E-mail}{mobilephone}{homepage}
% be careful of _ in emaill address
% {E-mail}{mobilephone}
% keep the last empty braces!
%\contactInfo{xxx@yuanbin.me}{(+86) 131-221-87xxx}{}
 
\section{目前职位}
\textbf{助理研究员},影像数据中心,上海交通大学医学院附属精神卫生中心 (2022.9-至今)\textbf{}\\
1. 基于共识推断和因果推断的强迫症小脑影像标志物研究;\\
2. 正念干预精神分裂症起效的因果效应与机制研究。

%% Current research interests
\section{研究兴趣}
{\bf 神经数据科学与神经信息学}\\
1. {\bf 因果推断和因果学习}及其应用( 统计与人工智能前沿);\\
2. {\bf 神经信息学(neuroinformatics)}(大数据与平台能力);\\
3. {\bf 多源异构数据建模与分析}(神经影像:s/f/dMRI,i/M/EEG/LFP,行为数据,临床评估数据,移动和传感数据等,神经和精神疾病的分类与进展建模(大数据建模与计算能力)。 \\

%% Education
\section{教育背景}
{\bf 北京邮电大学} \hfill {\em 2012年9月 - 2017年1月} \\
\textbf{博士},信息与通信系统工程 \hfill {田辉 教授}\\
网络与交换国家重点实验室\\
{\bf 北京邮电大学} \hfill {\em 2010年9月 - 2012年6月} \\
\textbf{硕士},电路与系统 \hfill {范春晓 教授}\\
无线新技术教育部重点实验室\\
{\bf 中南大学} \hfill {\em 2006年6月 - 2010年6月} \\
\textbf{本科},电子信息科学与技术 \hfill {许雪梅 教授}

%% Research Exp
\section{研究经历}
\datedsubsection{\textbf{1.博士后, 开放和可重复神经数据科学, 蒙特利尔神经所(MNI),麦吉尔大学\\}}{导师:\textbf{Jean-Baptiste Poline和Abbas Sadikot}, \textbf{2019.11 - 2022.9}}
\begin{itemize}
  \item \textbf{晚期特发性震颤小脑影像标志物的可靠性研究},研究采用预注册加二轮评审模式,是\emph{目前研究该问题规模最大、方法最严谨、统计效力最高的研究工作};
  \item \textbf{神经信息学与标准影像处理工作流},协助维护MNI数据镜像(2PB+),我在该项目中构建的影像处理流程已被原实验室拓展和泛化为\emph{领域影像标准工作流}进行领域推广(\href{https://github.com/neurodatascience/nipoppy}{\bf{Nipoppy}}); 
  \item \textbf{协助完成麦吉尔量化生命科学课程} \href{https://neurodatascience.github.io/QLS612-Overview/}{QLSC612: 神经数据科学}。
\end{itemize}

\datedsubsection{\textbf{2.博士后, 脑连接估计和对比, 中古联合实验室, 电子科技大学\\}}{导师:\textbf{Pedro Valdes-Sosa}(外专千人,\textbf{2024国际科学技术合作奖}) , \textbf{2017.3 - 2019.10}}
\begin{itemize}
  \item \textbf{基于猕猴同步脑电与皮质脑电估计和对比源功能连接},完成同步脑电和皮质脑电源连接对比平台并将代码开源;
  \item \textbf{建设、维护和管理研究中心计算平台}(1000+核,500T存储), 部署CBRAIN计算平台与LORIS存储平台; 
  \item 参加2018年OHBM hackathon并\textbf{讲授Git/Github课程}。
\end{itemize}

\datedsubsection{\textbf{3.研究助理, 计算教育学, 大数据研究中心, 电子科技大学\\}}{导师:\textbf{周涛}(中心主任,成都数之联科技有限公司创始人) , \textbf{2015.6 - 2018.9}}
\begin{itemize}
  \item 计算教育学,基于匿名学生卡数据(电子记录数据)和量表数据, 量化学生性格(大五人格),基于此提高对学生成绩的预测精度;
  \item 量化研究学生社会经济地位对其学业表现的影响。
\end{itemize}

\datedsubsection{\textbf{4.博士生, 计算社会学, 网络与交换国家重点实验室, 北京邮电大学\\}}{导师:\textbf{田辉}(研究室主任) , \textbf{2012.9 - 2017.3}}
\begin{itemize}
  \item 基于通信数据研究个人中心网络(ECN),发现了个人中心网络的新圈层及其特性(\textbf{成果被《科学年鉴》相关章节引用});
  \item 开发了egoPortray可视分析系统 \textbf{(MMM2017, CCF-A会议)};
  \item 提出了基于云片的多终端协同内容分发网络边缘计算构架。
\end{itemize}

\section{教学经验}
% increase linespacing [parsep=0.5ex]
\datedsubsection{1. \textbf{《机器学习及其在神经影像中的应用》},上海交通大学,\href{https://github.com/Tutorial-series/nilearn\_course}{nilearn\_course},2024;}{}
\datedsubsection{2. \href{https://github.com/Vincent-wq/causal_course_eeg}{\textbf{《Causality and mediation tools for q-EEG and clinical applications》}},\textbf{OHBM 2024}《Global Open Science Electrophysiology》,co-organizer,2024;}{}
\datedsubsection{3. \href{https://neurodatascience.github.io/QLS612-Overview/}{\bf{《QLSC612: 神经数据科学》}}, 麦吉尔大学, 2021-2022;}{}
\datedsubsection{4. \href{https://github.com/Tutorial-series/git_github_4OHBMhackthon2018_qingwang}{\bf{《Git/Github for beginners》}}, OHBM 2018 - brainhack, 2021-2022。}{}
\datedsubsection{5. \href{https://github.com/Tutorial-series/fmriprep_vincent_2020}{\bf{《fMRIPrep教程: 实用指南》}}, 2020-2021;}{}

\section{基金与奖励}
% increase linespacing [parsep=0.5ex]
\datedsubsection{1. 《连续θ爆发序列刺激治疗强迫症起效的动态计算机制研究》,23zd01,上海市精神卫生中心院级重点项目,30万元人民币,2024-2027;}{}
\datedsubsection{2. 《晚期特发性震颤小脑影像标志物的可靠性研究》,获麦吉尔计算医学项目\href{https://www.mcgill.ca/micm/programs/micm-researchmatch-0/researchmatch-v3-joint-projects}{MiCM Research Match V3.0}和\href{https://www.mcgill.ca/micm/programs/micm-researchmatch-0/researchmatch-v4-results}{V4.0}两期资助,麦吉尔大学,共计21,000加元,2020-2022。}{}

% \section{\faInfo\ 社会实践/其他}
\section{学术服务}
\datedsubsection{1. \href{https://ossig.netlify.app/\#OSSIG_team}{OHBM (脑图谱领域顶级国际会议) Hackathon \textbf{联席主席}(2023, 2024)};}{}
\datedsubsection{2. \href{https://open-sci.cn/index.html\#!}{开放科学中文社区}委员会资深委员;}{}
\datedsubsection{3. 组织\href{https://opensci-cn.github.io/cogscimeetup2023/}{COGSCI 2023 MEETUPS Shanghai: Computational Cognitive Neuroscience and Computational Psychiatry} (co-organizer);}{}
\datedsubsection{4. 受国际最大开放科学平台OSF邀请,分享在中国推进开放科学的经验,\href{https://www.youtube.com/watch?v=1jDOTdbK0zY}{YouTube回放};}{}
\datedsubsection{5. 组织COSN Summer Hackathon 2023暨“心与脑”国际合作学术交流会(国内第二次hackathon,安徽医科大承办),\href{https://mp.weixin.qq.com/s/C2E1RlK7FGKXtgWOG8tQFQ}{微信文章};}{}
\datedsubsection{6. 审稿人:Imaging Neuroscience, NeuroImage, Brain Topography, PLoS Computational Biology, PLoS ONE, Frontiers in Neuroinformatics, OHBM annual meeting, European Physics Letters and \textit{etc.};}{}

\section{期刊论文}
(1) Mou Xinyu, Cuilin He, Liwei Tan, ...,\textbf{Qing Wang}, et al. ‘ChineseEEG: A Chinese Linguistic Corpora EEG Dataset for Semantic Alignment and Neural Decoding’. Scientific Data 11, no. 1 (29 May 2024): 550. \\
(2)\textbf{Qing Wang}*, Qiao Wang, Ru-yuan Zhang. Claim Causality with Clarity.Psychoradiology 3 (1 March 2023): kkad007. \\
(3) Haiyang Jin\#,\textbf{Qing Wang}\#,..., Xinian Zuo, Chuanpeng Hu*. The Chinese Open Science Network (COSN): Building an Open Science community from scratch, Advances in Methods and Practices in Psychological Science 6, no. 1 (27 March 2023): 1–17. (\textbf{IF=15.8})\\
(4) \textbf{Qing Wang}\#, Meshal Aljassar1\#,Nikhil Bhagwat,Yashar Zeighami,Alan C Evans,Alain Dagher,G.Bruce Pike,Abbas F. Sadikot*,Jean-BaptistePoline*. Reproducibility of Cerebellar involvement as quantified by consensus structural MRI biomarkers in Advanced Essential Tremor, Scientific Reports 13, no. 1 (11 January 2023): 581.\\
(5) Ren Peng*, Sunpei Huang, Yukun Feng, Jinying Chen, \textbf{Qing Wang}, et al. Assessment of Balance Control Subsystems by Artificial Intelligence. IEEE Transactions on Neural Systems and Rehabilitation Engineering, no. 28.3(2020):658-668.\\
(6) \textbf{Qing Wang}, Pedro A Vald{\'e}s-Hernandez, Jorge Bosch-Bayard, Naoya Oosugi, Misako Komatsu, Naotaka Fujii, and Pedro A Vald{\'e}s-Sosa*. EECoG-Comp: An Open Source Platform for Concurrent EEG ECoG Comparisons. Brain Topogr., no. 32(2019): 550-568.\\
(7) Ren Peng*, Shiang Hu, Zhenfeng Han, \textbf{Qing Wang}, et al. Movement Symmetry Assessment by Bilateral Motion Data Fusion. IEEE Transactions on Biomedical Engineering 66, no. 1 (2018): 225-236.\\
(8) Kiar Gregory, Robert J Anderson, ... , Pedro A Vald{\'e}s-Sosa, \textbf{Qing Wang}, ... et al. NeuroStorm: Accelerating Brain Science Discovery in the Cloud. arXiv: 1803.03367 (2018). \\
(9) Cao, Yi, Jian Gao, Defu Lian, Zhihai Rong, Jiatu Shi, \textbf{Qing Wang}, Yifan Wu, Huaxiu Yao, and Tao Zhou*. Orderliness predicts academic performance: behavioural analysis on campus lifestyle. Journal of The Royal Society Interface 15, no. 146 (2018): 20180210.\\
(10) \textbf{Qing Wang}, Jian Gao, Tao Zhou*, Zheng Hu, and Hui Tian*. Critical size of ego communication networks. EPL (Europhysics Letters) 114, no. 5 (2016): 58004.\\
(11) \textbf{Qing Wang}, Hu Zheng*, Wang Ming, and Liu Haifeng. Cactse: Cloudlet aided cooperative terminals service environment for mobile proximity content delivery. China Communications 10, no. 6 (2013): 47-59.\\
(12) Zhou Lichao, Xu Xuemei*, Li An and \textbf{Qing Wang}. System of Home Monitoring and Controlling Based on ZigBee and Infrared [J]. Modern Electronics Technique 17 (2010).\\

\section{会议论文}
(13) \textbf{Qing Wang}, Pedro A Valdes-Hernandez, and Pedro A Vald{\'e}s-Sosa*. S129. An open source platform for concurrent EEG ECoG comparisons. Clinical Neurophysiology 129 (2018): e190.\\
(14) \textbf{Qing Wang}*, Jiansu Pu, Yuanfang Guo, Zheng Hu, and Hui Tian. egoPortray: Visual Exploration of Mobile Communication Signature from Egocentric Network Perspective. In International Conference on Multimedia Modeling, pp. 649-661. Springer, Cham, 2017. (\textbf{CCF-A})\\
(15) Han Mei, \textbf{Qing Wang}, Rao Yunbo, and Jiansu Pu*. egoStellar: Visual Analysis of Anomalous Communication Behaviors from Egocentric Perspective. In Computer and Information Science Communications 2018.\\

%% Techni
\section{技术能力}
% increase linespacing [parsep=0.5ex]
\begin{itemize}[parsep=0.2ex]
  \item \textbf{神经影像处理}: MEEG: MNE, BrainStorm, EEGLab, FieldTrip;MRI: fMRIPrep, SPM, FSL, Freesurfer, ANTs(itksnap), Nipy, Nilearn, heudiconv, fMRIPrep, MRIQC, TracoFlow, Dipy, etc.;
  \item \textbf{大数据计算}: HDF5, HDFS (distributed storage), Hadoop\&Spark, CBRAIN(computing), LORIS (database), Compute Canada (Globus), CONP and SLURM, singularity, docker et al.
  \item \textbf{信息可视化}: D3js, eChart, and python libraries like seaborn, ggplot, Bokeh, and matplotlib, et al.
  \item \textbf{通用编程与工具}: Linux and windows programming environments, Python(numpy, scipy, pandas, statsmodels, scikit-learn, pymc3, matplotlib, pyTorch, etc); Matlab, R, scala, JS, C++,Git/GitHub, Office365, Adobe AI,PS, et al.
\end{itemize}

%% Techni
\section{推荐人}
% increase linespacing [parsep=0.5ex]
\datedsubsection{\textbf{Jean-Baptiste Poline教授}(博士后导师),神经内科和神经外科,麦吉尔大学;Neurohub平台主席,加拿大开放神经科学平台(CONP)技术委员会主席;Ludmer神经信息与心理健康中心PI}{}
\textbf{邮件}: \href{jean-baptiste.poline@mcgill.ca}{jean-baptiste.poline@mcgill.ca}\\
主页: \href{https://www.mcgill.ca/neuro/jean-baptiste-poline-phd}{https://www.mcgill.ca/neuro/jean-baptiste-poline-phd}\\

\datedsubsection{\textbf{Pedro A Vald{\'e}s-Sosa教授}|(博士后导师),外专千人,2024国际科学技术合作奖,拉丁美洲科学院院士,中国-古巴神经科技转化前沿研究联合实验室主任,电子科技大学}{}
\textbf{邮件}: \href{pedro.valdes@neuroinformatics-collaboratory.org}{pedro.valdes@neuroinformatics-collaboratory.org}\\
主页: \href{https://en.wikipedia.org/wiki/Pedro\_Vald\%C3\%A9s}{https://en.wikipedia.org/wiki/Pedro\_Vald\%C3\%A9s}\\

\datedsubsection{\textbf{Alan C Evans教授}(主要合作者),加拿大皇家科学院院士,神经内科和神经外科,James McGill教授,McConnell影像中心,Ludmer神经信息与心理健康中心联合主任,麦吉尔大学}{}
\textbf{邮件}: \href{alan@bic.mni.mcgill.ca}{alan@bic.mni.mcgill.ca}\\
主页: \href{https://www.mcgill.ca/neuro/research/researchers/evans}{https://www.mcgill.ca/neuro/research/researchers/evans}\\

%% Reference
%\newpage
%\bibliographystyle{IEEETran}
%\bibliography{mycite}
\end{document}
